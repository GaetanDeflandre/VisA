% --------------------------------------
% Document Class
% --------------------------------------
\documentclass[a4paper,11pt]{article}
% --------------------------------------



% --------------------------------------
% Use Package
% --------------------------------------

\usepackage[francais]{babel}
\usepackage{ucs}
\usepackage[utf8]{inputenc}
\usepackage[T1]{fontenc}

\usepackage{makeidx}
\usepackage{color}
\usepackage{graphicx}
\usepackage{float}
\usepackage[hidelinks]{hyperref} 
\usepackage{geometry}
%\usepackage{lastpage}
%\usepackage{marginnote}
\usepackage{fancyhdr}
%\usepackage{titlesec}
%\usepackage{framed}
\usepackage{amsmath}
\usepackage{empheq}
\usepackage{array}
\usepackage{multicol}
\usepackage{csquotes}
%\usepackage{adjustbox}

% insert code
\usepackage{listings}

% define our color
\usepackage{xcolor}

% code color
\definecolor{ligthyellow}{RGB}{250,247,220}
\definecolor{darkblue}{RGB}{5,10,85}
\definecolor{ligthblue}{RGB}{1,147,128}
\definecolor{darkgreen}{RGB}{8,120,51}
\definecolor{darkred}{RGB}{160,0,0}

% other color
\definecolor{ivi}{RGB}{141,107,185}

\def\verticaltext#1{\rotatebox[origin=c]{90}{\x{#1}}}


\lstset{
    language=python,
    captionpos=b,
    extendedchars=true,
    frame=lines,
    numbers=left,
    numberstyle=\tiny,
    numbersep=5pt,
    keepspaces=true,
    breaklines=true,
    showspaces=false,
    showstringspaces=false,
    breakatwhitespace=false,
    stepnumber=1,
    showtabs=false,
    tabsize=3,
    basicstyle=\small\ttfamily,
    backgroundcolor=\color{ligthyellow},
    keywordstyle=\color{ligthblue},
    morekeywords={include, printf, uchar},
    identifierstyle=\color{darkblue},
    commentstyle=\color{darkgreen},
    stringstyle=\color{darkred},
}


% --------------------------------------



% --------------------------------------
% Page setting
% --------------------------------------
%\pagestyle{empty}
\setlength{\headheight}{15pt}

\setcounter{secnumdepth}{3}
\setcounter{tocdepth}{2}

\makeatletter
\@addtoreset{chapter}{part}
\makeatother 

\hypersetup{       % parametrage des hyperliens
  colorlinks=true,  % colorise les liens
  breaklinks=true,  % permet les retours à la ligne pour les liens 
                    % trop longs
  urlcolor= blue,   % couleur des hyperliens
  linkcolor= black, % couleur des liens internes aux documents 
                    % (index, figures, tableaux, equations,...)
  citecolor= green  % couleur des liens vers les references 
                    % bibliographiques
}

% --------------------------------------

% --------------------------------------
% Information
% --------------------------------------
\title{
  \noindent\hrulefill \\
  \vspace{10mm}
  \textbf{Compte-rendu VisA} \\
  \vspace{5mm}
  TP: Approche de la logique floue.
}

\author{Gaëtan DEFLANDRE}
% --------------------------------------

\definecolor{myColor}{rgb}{0.5, 0.1, 0.75}

% --------------------------------------
% Begin content
% --------------------------------------
\begin{document}

\maketitle
\noindent\hrulefill \\


\section*{Introduction}

Dans ces TPs, nous poursuivons l'étude de la logique floue. Nous allons 
utiliser les principes de cette logique pour segmenter des images. \\

Durant ces TPs, nous implémentons en logique floue des algorithmes 
classifications de classes dans les images. En logique booléenne, 
certaines de ces méthodes sont très utilisés, telles que le C-Means. Il 
existe la variante floue de cette méthode, elle s'apelle FCM (Fuzzy 
C-Means). Ainsi, nous verrons différents algorithmes de classification 
floue.\\


\newpage



Lors de ces TPs, nous utilisons pour image de référence, une image 
contenant six classes. Cette image est la suivante:

\begin{figure}[H]
  \begin{center} 
    \includegraphics[width=180px]{../img/6_classes_RGB.png}
    \caption{\texttt{6\_classes\_RGB.tif} utilisé pour la classification.}
  \end{center}
\end{figure}

\section{Algorithme Fuzzy C-Means}

\subsection{Initialisation}

\begin{itemize}
 \item Choisir le nombre de classes.
 \item m = 2 par défaut
 \item Initialisation des centroides aléatoire.
 \item Initialisation des distance $x_i-c_k$\\
\end{itemize}

Initialisation de la matrice de degrés d'appartenance, de patition $U$\\

\begin{Large}
$$
u_{ik} = \frac{1}{\sum\limits_{j=1}(\frac{d_{ik}}{d_{ij}})^{\frac{2}{m-1}}}
$$
\end{Large}

\subsection{Résultats}

\begin{tabular}{|l|c|}
  \hline
  Nombre de classes & 6 \\
  \hline
  Valeur de m & 2 \\
  \hline
  Nombre d'itération & 100 \\
  \hline
  Valeur de seuil de  & 0.000001 \\
  \hline
  Randomisation & 1 \\
  \hline
\end{tabular}

\begin{figure}[H]
  \begin{center} 
    \includegraphics[width=180px]{../img/segFCM.png}
    \caption{Segmentation avec la méthode FCM}
  \end{center}
\end{figure}

\section{Algorithme Hard C-Means}

\subsection{Résultats}

\begin{tabular}{|l|c|}
  \hline
  Nombre de classes & 6 \\
  \hline
  Valeur de m & 2 \\
  \hline
  Nombre d'itération & 1000 \\
  \hline
  Valeur de seuil de  & 0.0000001 \\
  \hline
  Randomisation & 1 \\
  \hline
\end{tabular}

\begin{figure}[H]
  \begin{center} 
    \includegraphics[width=180px]{../img/segHCM.png}
    \caption{}
  \end{center}
\end{figure}

\section{Algorithme Possibilistic C-Means}

\section{Algorithme de Davé}

\subsection{Résultats}

\begin{tabular}{|l|c|}
  \hline
  Nombre de classes & 6 \\
  \hline
  Valeur de m & 2 \\
  \hline
  Nombre d'itération & 100 \\
  \hline
  Valeur de seuil de  & 0.000001 \\
  \hline
  Randomisation & 1 \\
  \hline
\end{tabular}

\begin{figure}[H]
  \begin{center} 
    %\includegraphics[width=180px]{../img/segFCM.png}
    \caption{}
  \end{center}
\end{figure}


\end{document}
