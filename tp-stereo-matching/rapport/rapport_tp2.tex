% --------------------------------------
% Document Class
% --------------------------------------
\documentclass[a4paper,11pt]{article}
% --------------------------------------



% --------------------------------------
% Use Package
% --------------------------------------


%\usepackage[francais]{babel}
%\usepackage{ucs}
\usepackage[utf8]{inputenc}
\usepackage[T1]{fontenc}

\usepackage{makeidx}
\usepackage{color}
\usepackage{graphicx}
\usepackage{float}
\usepackage[hidelinks]{hyperref} 
\usepackage{geometry}
%\usepackage{lastpage}
%\usepackage{marginnote}
\usepackage{fancyhdr}
%\usepackage{titlesec}
%\usepackage{framed}
\usepackage{amsmath}
\usepackage{empheq}
\usepackage{array}
\usepackage{multicol}
\usepackage{csquotes}
%\usepackage{adjustbox}

% insert code
\usepackage{listings}

% define our color
\usepackage{xcolor}

% code color
\definecolor{ligthyellow}{RGB}{250,247,220}
\definecolor{darkblue}{RGB}{5,10,85}
\definecolor{ligthblue}{RGB}{1,147,128}
\definecolor{darkgreen}{RGB}{8,120,51}
\definecolor{darkred}{RGB}{160,0,0}

% other color
\definecolor{ivi}{RGB}{141,107,185}


\lstset{
    language=C++,
    captionpos=b,
    extendedchars=true,
    frame=lines,
    numbers=left,
    numberstyle=\tiny,
    numbersep=5pt,
    keepspaces=true,
    breaklines=true,
    showspaces=false,
    showstringspaces=false,
    breakatwhitespace=false,
    stepnumber=1,
    showtabs=false,
    tabsize=3,
    basicstyle=\small\ttfamily,
    backgroundcolor=\color{ligthyellow},
    keywordstyle=\color{ligthblue},
    morekeywords={include, printf, uchar},
    identifierstyle=\color{darkblue},
    commentstyle=\color{darkgreen},
    stringstyle=\color{darkred},
}


% --------------------------------------



% --------------------------------------
% Page setting
% --------------------------------------
%\pagestyle{empty}
\setlength{\headheight}{15pt}

\setcounter{secnumdepth}{3}
\setcounter{tocdepth}{2}

\makeatletter
\@addtoreset{chapter}{part}
\makeatother 

\hypersetup{         % parametrage des hyperliens
  colorlinks=true,      % colorise les liens
  breaklinks=true,      % permet les retours à la ligne pour les liens trop longs
  urlcolor= blue,       % couleur des hyperliens
  linkcolor= black,     % couleur des liens internes aux documents (index, figures, tableaux, equations,...)
  citecolor= green      % couleur des liens vers les references bibliographiques
}

% --------------------------------------

% --------------------------------------
% Information
% --------------------------------------
\title{
  \noindent\hrulefill \\
  \vspace{10mm} Compte-rendu TP2 VisA: Mise en correspondance stéréoscopique
}

\author{Gaëtan DEFLANDRE}
% --------------------------------------

\definecolor{myColor}{rgb}{0.5, 0.1, 0.75}

% --------------------------------------
% Begin content
% --------------------------------------
\begin{document}


\maketitle

\noindent\hrulefill \\


\section{Introduction}
Il existe plusieurs techniques de reconstruction 3D, la stéréovision fait partie de ces techniques.
La stéréoscopie consiste à avoir deux projections perspectives donnant une géométrie épipolaire. 
Avec ces deux images et les propriétés de la géométrie épipolaire, il est possible d'effectuer une 
mise en correspondance stéréoscopique, qui est faite dans ce TP. Cette mise en correspondance nous 
permet de retrouver la profondeur.\\

\newpage

\section{Calcul de la matrice fondamentale}

En stéréovision les deux projections nous donnent deux images que nous appelons image de gauche et 
image de droite. Il existe une matrice que l'on nomme matrice fondamentale, elle caractèrise la 
géométrie épipolaire liée aux deux projections. Il est nécessaire de calculer cette matrice pour 
effectuer la mise en correspondance des deux images.\\


Formule de la matrice fondamentale:
$$
F = (P_2 O_1)^{\times} P_2 P_1^+
$$

avec le produit vectoriel:
$$
p^{\times} = 
\begin{pmatrix}
  0 && -p_z && p_y \\
  p_z && 0 && -p_x \\
  -p_y && p_x && 0 \\
\end{pmatrix}
$$

et $P^+$, la pseudo-inverse de la matrice $P$.\\


\begin{lstlisting}[caption=Fontion qui calcul le produit vectoriel]
/// \brief Initialise une matrice de produit vectoriel.
///
/// @param v: vecteur colonne (3 coordonnees)
/// @return matrice de produit vectoriel
Mat iviVectorProductMatrix(const Mat& v) {

    Mat mVectorProduct = Mat::eye(3, 3, CV_64F);

    mVectorProduct.at<double>(0,0) = 0.0;
    mVectorProduct.at<double>(1,0) = -v.at<double>(Point(2,0));
    mVectorProduct.at<double>(2,0) = v.at<double>(Point(1,0));
    mVectorProduct.at<double>(0,1) = v.at<double>(Point(2,0));
    mVectorProduct.at<double>(1,1) = 0.0;
    mVectorProduct.at<double>(2,1) = -v.at<double>(Point(0,0));
    mVectorProduct.at<double>(0,2) = -v.at<double>(Point(1,0));
    mVectorProduct.at<double>(1,2) = v.at<double>(Point(0,0));
    mVectorProduct.at<double>(2,2) = 0.0;

    // Retour de la matrice
    return mVectorProduct;
}
\end{lstlisting}

\begin{lstlisting}[caption=Fontion qui calcul la matrice fondamentale]
/// \brief Initialise et calcule la matrice fondamentale.
///
/// @param mLeftIntrinsic: matrice intrinseque de la camera gauche
/// @param mLeftExtrinsic: matrice extrinseque de la camera gauche
/// @param mRightIntrinsic: matrice intrinseque de la camera droite
/// @param mRightExtrinsic: matrice extrinseque de la camera droite
/// @return matrice fondamentale
Mat iviFundamentalMatrix(const Mat& mLeftIntrinsic,
                         const Mat& mLeftExtrinsic,
                         const Mat& mRightIntrinsic,
                         const Mat& mRightExtrinsic) {

    // Doit utiliser la fonction iviVectorProductMatrix
    Mat mFundamental = Mat::eye(3, 3, CV_64F);

    Mat reduc = (Mat_<double>(3,4) <<
        1.0, 0.0, 0.0, 0.0,
        0.0, 1.0, 0.0, 0.0,
        0.0, 0.0, 1.0, 0.0
        );
    Mat p1 = mLeftIntrinsic * reduc * mLeftExtrinsic;
    Mat p2 = mRightIntrinsic * reduc * mRightExtrinsic;

    Mat iE1 = mLeftExtrinsic.inv();
    Mat o1 = iE1.col(3);

    mFundamental = iviVectorProductMatrix(p2 * o1) * p2 * p1.inv(DECOMP_SVD);

    // Retour de la matrice fondamentale
    return mFundamental;
}
\end{lstlisting}



\section{Conclusion}



\section{Annexe}



\end{document}
