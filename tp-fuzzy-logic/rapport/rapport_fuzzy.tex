% --------------------------------------
% Document Class
% --------------------------------------
\documentclass[a4paper,11pt]{article}
% --------------------------------------



% --------------------------------------
% Use Package
% --------------------------------------

\usepackage[francais]{babel}
\usepackage{ucs}
\usepackage[utf8]{inputenc}
\usepackage[T1]{fontenc}

\usepackage{makeidx}
\usepackage{color}
\usepackage{graphicx}
\usepackage{float}
\usepackage[hidelinks]{hyperref} 
\usepackage{geometry}
%\usepackage{lastpage}
%\usepackage{marginnote}
\usepackage{fancyhdr}
%\usepackage{titlesec}
%\usepackage{framed}
\usepackage{amsmath}
\usepackage{empheq}
\usepackage{array}
\usepackage{multicol}
\usepackage{csquotes}
%\usepackage{adjustbox}

% insert code
\usepackage{listings}

% define our color
\usepackage{xcolor}

% code color
\definecolor{ligthyellow}{RGB}{250,247,220}
\definecolor{darkblue}{RGB}{5,10,85}
\definecolor{ligthblue}{RGB}{1,147,128}
\definecolor{darkgreen}{RGB}{8,120,51}
\definecolor{darkred}{RGB}{160,0,0}

% other color
\definecolor{ivi}{RGB}{141,107,185}

\def\verticaltext#1{\rotatebox[origin=c]{90}{\x{#1}}}


\lstset{
    language=python,
    captionpos=b,
    extendedchars=true,
    frame=lines,
    numbers=left,
    numberstyle=\tiny,
    numbersep=5pt,
    keepspaces=true,
    breaklines=true,
    showspaces=false,
    showstringspaces=false,
    breakatwhitespace=false,
    stepnumber=1,
    showtabs=false,
    tabsize=3,
    basicstyle=\small\ttfamily,
    backgroundcolor=\color{ligthyellow},
    keywordstyle=\color{ligthblue},
    morekeywords={include, printf, uchar},
    identifierstyle=\color{darkblue},
    commentstyle=\color{darkgreen},
    stringstyle=\color{darkred},
}


% --------------------------------------



% --------------------------------------
% Page setting
% --------------------------------------
%\pagestyle{empty}
\setlength{\headheight}{15pt}

\setcounter{secnumdepth}{3}
\setcounter{tocdepth}{2}

\makeatletter
\@addtoreset{chapter}{part}
\makeatother 

\hypersetup{       % parametrage des hyperliens
  colorlinks=true,  % colorise les liens
  breaklinks=true,  % permet les retours à la ligne pour les liens 
                    % trop longs
  urlcolor= blue,   % couleur des hyperliens
  linkcolor= black, % couleur des liens internes aux documents 
                    % (index, figures, tableaux, equations,...)
  citecolor= green  % couleur des liens vers les references 
                    % bibliographiques
}

% --------------------------------------

% --------------------------------------
% Information
% --------------------------------------
\title{
  \noindent\hrulefill \\
  \vspace{10mm}
  \textbf{Compte-rendu VisA} \\
  \vspace{5mm}
  TP: Approche de la logique floue.
}

\author{Gaëtan DEFLANDRE}
% --------------------------------------

\definecolor{myColor}{rgb}{0.5, 0.1, 0.75}

% --------------------------------------
% Begin content
% --------------------------------------
\begin{document}

\maketitle
\noindent\hrulefill \\


\section*{Introduction}

Les techniques de segmentation comme le clustering permettent de 
calculer des classes d'appartenance. Ces classes permettent de 
décrire les pixels de l'image de manière net et sans ambiguité.\\

Or, ce type d'apporche n'est pas toujours évident. Dans cette 
configuration, un pixel appartient ou pas à une classe. La logique 
floue permetter d'éviter ce genre de segmentation trop net, avec 
fonction d'appartence qui donnent des facteur plutôt qu'une valeur 
booléenne.\\


\newpage


\section{Fonctions d'appartenance}

Soit les fonctions d'appartenance flous aux température basse, 
moyenne et haute.

\begin{figure}[H]
  \begin{center}
  \includegraphics[height=280px]{images/exercice1.png}
  \caption{Les fonctions d'appartenance.}
  \end{center}
\end{figure}

Voici, le tableau qui montre les degrés d'appartenance aux différentes 
fonctions pour une température de 16\degre C.

\begin{table}[H]
  \caption{Degrés d'appartenance pour une température de 16\degre C}

  \begin{center}
    \begin{tabular}{|l|c|}
      \hline
       & Degrés d'appartenance \\
      \hline
      \hline
      Temp basse & 0.4 \\
      \hline
      Temp moyenne & 0.6 \\
      \hline
      Temp elevée & 0.0\\
      \hline
    \end{tabular}
  \end{center}
\end{table}


\section{Opérateurs de la logique floue}

Maintenant, nous allons voir l'opérateur min qui donne le sous-ensemble 
flou minimum entre deux fonctions flous. Voici par exemple le 
minimum entre la fonction de température basse et moyenne.

\begin{figure}[H]
  \begin{center}
  \includegraphics[height=280px]{images/min.png}
  \caption{Fonction minimum entre la fonction de température basse et moyenne.}
  \end{center}
\end{figure}

Ensuite, l'opérateur max donne le sous-ensemble flou maximum 
entre deux fonctions flous. Par exemple le maximum entre la 
fonction de température moyenne et haute.

\begin{figure}[H]
  \begin{center}
  \includegraphics[height=280px]{images/max.png}
  \caption{Fonction maximum entre la fonction de température moyenne et haute.}
  \end{center}
\end{figure}



\section{Implication floue}

\begin{figure}[H]
  \begin{center}
  \includegraphics[height=170px]{images/low_mamdani_arrow.png}
  \caption{Fonction de température basse et implication floue.}
  \end{center}
\end{figure}




\section*{Conclusion}

\newpage





\section*{Conclusion}


\end{document}
